%%%%%%%%%%%%%%%%%%%%%%%%%%%%%%%%%%%%%%%%%%%%%%%%%%%%%%%%%%%%%
\chapter*{Introduction}
\addcontentsline{toc}{chapter}{Introduction}
   The optimization of time-varying well settings,
  such as injection and production rates or bottom-hole pressure, is an
  important aspect of optimal reservoir management. Both gradient-based and
  derivative-free methods have been considered for this problem, and both are
  applicable in different situations. When the simulator source code is
  accessible, a gradient-based optimization method, in which the gradient is
  computed using an adjoint formulation, is often the method of choice since it
  is generally the most efficient.

In this paper we implement an adjoint formulation for compositional reservoir
simulation problems. Procedures of this type entail the application of optimal
control theory and have their roots in the calculus of variations
\citep{Bryson:1975,Stengel:1986}. Adjoint-based optimization techniques have been
used in a reservoir simulation setting both for history matching (see, e.g.,
  \citep{Gavalas,Chavent,Li,Oliver,Pallav:2006}) and for production optimization.
Much of the early work on their use for optimization of oil recovery was performed by
Ramirez and coworkers, who considered the optimization of several different
enhanced oil recovery (EOR) processes
\citep{Ramirez:book,Ramirez:1989,Ramirez:1993}. In subsequent work, the focus was
on gradient-based optimization (and in some cases on the optimization of `smart
  wells') for water flooding
\citep{Asheim,Virnovski,Sudaryanto:2000,Brouwer:2004,Pallav:2006}. Recent studies
have addressed the implementation of adjoint-based procedures into general
purpose simulators, the treatment of general constraints, and regularization and
other numerical issues \citep{Pallav:2008,Brouwer:2008,Doublet:2009,CPRA}. Refer
to \citep{Jansen:2011} for a more complete overview of adjoint-based optimization
methods. We note additionally that, although not considered here, derivative-free
methods can also be applied for production optimization problems -- see
\citep{echeverria:2011} for discussion and examples.

Although much of the early (1980s) work noted above focused on the application
of adjoint procedures for EOR problems, there has not been much work on the use
of adjoint techniques for large-scale (practical) compositional reservoir
simulation problems. This is likely due to the complexity entailed in
implementing adjoint procedures into a general purpose compositional reservoir
simulator and to the challenging computational problems that must be solved to
perform the optimizations. Compositional simulation is inherently more
challenging than black-oil simulation because of the need to perform
phase-equilibrium (flash) calculations for all grid blocks at every iteration of every time step. Adjoint formulations are challenging to code
because they require analytical derivatives of many variables, and the
increased complexity of compositional simulators renders these derivatives
much more cumbersome to calculate than in the case of a black-oil simulator.


In this work, we implement an adjoint treatment for multicomponent oil-gas
compositional systems through use of a recently developed automatic
differentiation capability \citep{Younis:2010}. The application of automatic
differentiation in the context of Stanford's General Purpose Research Simulator
(AD-GPRS) \citep{Cao:Thesis}, a modular simulator with many advanced features,
enables us to construct a gradient-based optimization framework suitable for
use in compositional problems. Our formulation includes the treatment of
bound, linear and nonlinear constraints. Along these lines, we consider two
different treatments for the nonlinear constraints: a formal treatment within
the optimizer, and a heuristic approach, where bound constraints are treated in the optimization and nonlinear constraints are satisfied in the forward model.


Although the results we present are for a discrete adjoint formulation,
we have also developed a continuous adjoint formulation. In a discrete
implementation, the governing equations for the so-called adjoint
system are constructed based on the discretized-in-time forward model
equations. In continuous formulations, by contrast, the adjoint
equations are formed from the continuous forward model. Recent
formulations for production optimization have generally been based on
discrete formulations. Consistent with this, Brouwer and Jansen~\citep{Brouwer:2004}
reviewed previous work and concluded that the discrete adjoint method
was preferable.



Recently an adjoint treatment for multicomponent oil-gas
compositional systems was presented in~\citep{Kourounis2014}. 
The formulation included an extensive discussion on engineering
constraints that should usually be taken into account in realistic scenarios. 
These constraints appear either as bounds (box constraints) 
on the control variables or as inequality constraints on 
nonlinear functions of the controls and states of the underlying PDEs.
 Two treatments were proposed for the nondifferentiable constraints: a formal treatment within
the optimizer performing lumping for all wells and time steps, and a heuristic approach,
where bound constraints are treated in the optimization and nondifferentiable 
constraints are satisfied in the forward model. The investigation showed that although standard
lumping techniques perform well for simple academic problems, they fail 
to obtain optimal solutions better than the reference for realistic problems.
That result motivated further developments of formal constraint-handling techniques.
In ~\citep{Kourounis2015} the author introduces a new formal treatment for the nondifferentiable 
constraints where lumping is avoided to allow for a more realistic discretization
of the nonlinear constraints. The performance of the new approach is compared to
the ones introduced in~\citep{Kourounis2014} for several different examples
of increased complexity.



\endinput
%%% Local Variables: 
%%% mode: latex
%%% TeX-master: "ptyxiakn"
%%% End: 
